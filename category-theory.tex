\documentclass[]{article}
\usepackage{amsmath,amscd}
\usepackage{amssymb}
\usepackage{amsthm}
\usepackage{todonotes}
\usepackage{tikz-cd}
\usepackage{calligra}
\DeclareMathAlphabet{\catty}{T1}{calligra}{m}{n}

\usepackage [english]{babel}
\usepackage [autostyle, english = american]{csquotes}
\MakeOuterQuote{"}

\newtheorem{theorem}{Theorem}
\newtheorem{definition}{Definition}

\newcommand{\what}{}
% \newcommand{\cat}[1]{\mathbf{#1}}
\newcommand{\defn}[2]{
\renewcommand{\what}{\textit{#1} }
\textbf{Definition:} #2\\
}
\newcommand{\diag}[1]{$$\begin{CD}#1\end{CD}$$}
\newcommand{\cdr}[1]{\arrow[swap]{r}{#1}}
\newcommand{\cdl}[1]{\arrow{l}{#1}}
\newcommand{\cdd}[1]{\arrow{d}{#1}}
\newcommand{\cdu}[1]{\arrow{u}{#1}}

\newcommand{\tfarr}[3]{\ensuremath{#1 : #2 \to #3}}
\newcommand{\functor}[3]{\ensuremath{#1 : \cat{#2} \to \cat{#3}}}
\newcommand{\cat}[1]{\ensuremath{\!\! \catty{#1} \,\,}}

%opening
\title{Category Theory}
\author{Sandy Maguire}

\begin{document}

\maketitle

\begin{abstract}
My notes summarizing Awodey for the purposes of learning Category Theory. It's
going to be a great project.
\end{abstract}

\newpage

\section{Foundations}

\subsection{Definition of a Category}

A category \cat{C} consists of \textbf{objects} and \textbf{arrows} between
them. To be precise, every arrow has a domain and a codomain, both of which are
objects in the category. In addition, every object $X \in \cat{C}$ has an
\textbf{identity arrow} \tfarr{1_X}{X}{X}.

If \tfarr{f}{A}{B}, we say $dom(f) = A$ and $cod(f) = B$.

In addition, to be a category, \cat{C} must respect the following laws:

\begin{enumerate}
  \item{Composition: If \tfarr{f}{A}{B} and \tfarr{g}{B}{C}, there exists an
    arrow \tfarr{g \circ f}{A}{C}.}
  \item{Associative: $f \circ (g \circ h) = (f \circ g) \circ h$}.
  \item{Identity: $1_B \circ f = f = f \circ 1_A$, given \tfarr{f}{A}{B}}
\end{enumerate}

Anything that satisfies these laws is a category. It need not correspond to our
intuitions that "arrows are functions" or any such silliness.

\subsection{Definition of a Functor}

A functor \functor{F}{C}{D} is a mapping of objects in \cat{C} to objects in
\cat{D}, and likewise for arrows. A functor is a homomorphism across domains,
codomains and compositions.

That is to say:

\begin{enumerate}
  \item{$F(\tfarr{f}{A}{B}) = \tfarr{F(f)}{F(A)}{F(B)}$}
  \item{$F(g\circ f) = F(g) \circ F(f)$}
  \item{$F(1_A) = 1_{F(A)}$}
\end{enumerate}

There is an identity functor \functor{1_{\cat{C}}}{C}{C}, and because functors
compose, we have a category of categories: \cat{Cat}.

\subsection{Definition of an Isomorphism}

An arrow \tfarr{f}{A}{B} is called an isomorphism if there exists an arrow
\tfarr{g}{B}{A} such that $g \circ f = 1_A$ and $f \circ g = 1_B$.

\begin{theorem}
Isomorphisms are unique.
\begin{proof}
  For \tfarr{f}{A}{B} to be an isomorphism, we must have \tfarr{g}{B}{A}. Assume
  that the isomorphism is not unique, and thus that we also have
  \tfarr{g'}{B}{A}.

  By definition, we have $g \circ f = 1_A$. We can compose on both sides to get
  $g \circ f \circ g' = 1_A \circ g'$, but recall that $g'$ is an isomorphism,
  therefore $g \circ 1_B = 1_A \circ g'$. We can emit the identities, and thus
  $g = g'$.
\end{proof}
\end{theorem}

\begin{definition}
  A \textbf{group} is a single object category where every arrow is an
  isomorphism.
\end{definition}

\section{Constructions on Categories}

\subsection{Product Category}

The product of categories \cat{C} and \cat{D} is $\cat{C}\times\cat{D}$. It's
objects are the cartesian product of objects in \cat{C} and \cat{D}. Arrows are
likewise defined in this matter, with composition and units being defined
component-wise:

\begin{align*}
  1_{(C, D)} &= (1_C, \; 1_D) \\
  (f', g') \circ (f, g) &= (f' \circ f, \; g' \circ g)
\end{align*}

There are also projection functors
$\functor{\pi_1}{\cat{C}\times\cat{D}}{\cat{C}}$ and
$\functor{\pi_2}{\cat{C}\times\cat{D}}{\cat{C}}$ defined in the obvious way.

\subsection{Arrow Category}

The arrow category $\cat{C}^\rightarrow$ has objects which are arrows in \cat{C}
and its arrows are commutative squares in \cat{C}.

For example, given $f,\;f',\;g_1,\;g_2 \in \cat{C}$, there is an arrow $\tfarr{g
}{f}{f'} \in \cat{C}^\rightarrow$ such that $g = (g_1,\;g_2)$ if the follow
diagram exists in \cat{C}:

$$\begin{tikzcd}
  A \cdr{g_1} \cdd{f} & A' \cdd{f'} \\
  B \cdr{g_2} & B'
\end{tikzcd}$$

Composition acts as you'd expect. Given $h \circ g$, we get the diagram:

$$\begin{tikzcd}
  A \cdr{g_1} \cdd{f} & A' \cdr{h_1} \cdd{f'} & A'' \cdd{f''}  \\
  B \cdr{g_2} & B' \cdr{h_2} & B''
\end{tikzcd}$$

In other words, there is an arrow $f\to f'$ iff $g_2\circ f = f' \circ g_1$. But
what does this mean?

In a monoid, composition means concatenation, and thus over a free monoid of the
alphabet, $f = \text{art}$, $f' = \text{far}$, $g = (\text{t}, \text{f})$
because $\text{f}\circ\text{art} = \text{far}\circ\text{t}$. With the monoid of
addition over the natural numbers, the arrow category is complete; all objects
have infinite arrows between them because there are infinite ways of adding two
numbers to two numbers and having them equate.

\todo{more examples plz}
\todo{what are the functors?}

\subsection{Slice Category}

The slice category $\cat{C}/X$, given $X \in \cat{C}$ is a special case of the
arrow category when $B = B' = X$.

\end{document}
