\documentclass[]{article}
\usepackage{amsmath,amscd}
\usepackage{amssymb}
\usepackage{amsthm}
\usepackage{todonotes}
\usepackage{tikz-cd}
\usepackage{calligra}
\DeclareMathAlphabet{\catty}{T1}{calligra}{m}{n}

\usepackage [english]{babel}
\usepackage [autostyle, english = american]{csquotes}
\usepackage{fancyvrb}
\DefineVerbatimEnvironment{code}{Verbatim}{fontsize=\small}
\MakeOuterQuote{"}

\newtheorem{theorem}{Theorem}
\newtheorem{definition}{Definition}
\newtheorem{exercise}{Exercise}

\newcommand{\what}{}
% \newcommand{\cat}[1]{\mathbf{#1}}
\newcommand{\defn}[2]{
\renewcommand{\what}{\textit{#1} }
\textbf{Definition:} #2\\
}
\newcommand{\diag}[1]{$$\begin{CD}#1\end{CD}$$}
\newcommand{\cdr}[1]{\arrow[swap]{r}{#1}}
\newcommand{\cdrn}[1]{\arrow[swap,dotted]{r}{#1}}
\newcommand{\cdrr}[1]{\arrow[swap]{rr}{#1}}
\newcommand{\cddl}[1]{\arrow{dl}{#1}}
\newcommand{\cddr}[1]{\arrow[swap]{dr}{#1}}
\newcommand{\cdl}[1]{\arrow{l}{#1}}
\newcommand{\cdd}[1]{\arrow{d}{#1}}
\newcommand{\cdu}[1]{\arrow{u}{#1}}
\newcommand{\mono}{\rightarrowtail}
\newcommand{\epi}{\twoheadrightarrow}

\newcommand{\abs}[1]{\left|#1\right|}

\newcommand{\tfarr}[4][\to]{\ensuremath{#2 : #3 #1 #4}}
\newcommand{\functor}[3]{\ensuremath{#1 : \cat{#2} \to \cat{#3}}}
\newcommand{\cat}[1]{\ensuremath{\!\! \catty{#1} \,\,}}

%opening
\title{Category Theory}
\author{Sandy Maguire}

\begin{document}

\maketitle

\begin{abstract}
My notes summarizing Awodey for the purposes of learning Category Theory. It's
going to be a great project.
\end{abstract}

\newpage

\section{Foundations}

\subsection{Definition of a Category}

A category \cat{C} consists of \textbf{objects} and \textbf{arrows} between
them. To be precise, every arrow has a domain and a codomain, both of which are
objects in the category. In addition, every object $X \in \cat{C}$ has an
\textbf{identity arrow} \tfarr{1_X}{X}{X}.

If \tfarr{f}{A}{B}, we say $dom(f) = A$ and $cod(f) = B$.

In addition, to be a category, \cat{C} must respect the following laws:

\begin{enumerate}
  \item{Composition: If \tfarr{f}{A}{B} and \tfarr{g}{B}{C}, there exists an
    arrow \tfarr{g \circ f}{A}{C}.}
  \item{Associative: $f \circ (g \circ h) = (f \circ g) \circ h$}.
  \item{Identity: $1_B \circ f = f = f \circ 1_A$, given \tfarr{f}{A}{B}}
\end{enumerate}

Anything that satisfies these laws is a category. It need not correspond to our
intuitions that "arrows are functions" or any such silliness.

\subsection{Definition of a Functor}

A functor \functor{F}{C}{D} is a mapping of objects in \cat{C} to objects in
\cat{D}, and likewise for arrows. A functor is a homomorphism across domains,
codomains and compositions.

That is to say:

\begin{enumerate}
  \item{$F(\tfarr{f}{A}{B}) = \tfarr{F(f)}{F(A)}{F(B)}$}
  \item{$F(g\circ f) = F(g) \circ F(f)$}
  \item{$F(1_A) = 1_{F(A)}$}
\end{enumerate}

There is an identity functor \functor{1_{\cat{C}}}{C}{C}, and because functors
compose, we have a category of categories: \cat{Cat}.

\subsection{Definition of an Isomorphism}

An arrow \tfarr{f}{A}{B} is called an isomorphism if there exists an arrow
\tfarr{g}{B}{A} such that $g \circ f = 1_A$ and $f \circ g = 1_B$.

\begin{theorem}
Isomorphisms are unique.
\begin{proof}
  For \tfarr{f}{A}{B} to be an isomorphism, we must have \tfarr{g}{B}{A}. Assume
  that the isomorphism is not unique, and thus that we also have
  \tfarr{g'}{B}{A}.

  By definition, we have $g \circ f = 1_A$. We can compose on both sides to get
  $g \circ f \circ g' = 1_A \circ g'$, but recall that $g'$ is an isomorphism,
  therefore $g \circ 1_B = 1_A \circ g'$. We can emit the identities, and thus
  $g = g'$.
\end{proof}
\end{theorem}

\begin{definition}
  A \textbf{group} is a single object category where every arrow is an
  isomorphism.
\end{definition}

\begin{theorem}
Every group $G$ is isomorphic to a group of permutations.
\begin{proof}
  Define $f_g(x) = g \times x$ for $g \in G$. Since $G$ is a group, we also have
  $f_{g^{-1}}(x) = g^{-1} \times x = f_g^{-1}(x)$ which means $(f_g \circ
  f_g^{-1})(x) = (f_g^{-1} \circ f_g)(x) = x$. Therefore $f_g$ forms a group.

  Consider now a function $\tfarr{T}{G}{\bar{G}}$ where $T(g) = f_g$.  $T$ is a
  group homomorphism because $(f_g \circ f_h)(x) = f_g(f_h(x)) = f_g(h * x) = g
  * (h * x) = f_{g*h}(x)$.

  This is true for any $x$, therefore $T(g) \circ T(h) = f_g \circ f_h = f_{g*h}
  = T(g * h)$
\end{proof}
\end{theorem}

\section{Constructions on Categories}

\subsection{Product Category}

The product of categories \cat{C} and \cat{D} is $\cat{C}\times\cat{D}$. It's
objects are the cartesian product of objects in \cat{C} and \cat{D}. Arrows are
likewise defined in this matter, with composition and units being defined
component-wise:

\begin{align*}
  1_{(C, D)} &= (1_C, \; 1_D) \\
  (f', g') \circ (f, g) &= (f' \circ f, \; g' \circ g)
\end{align*}

There are also projection functors
$\functor{\pi_1}{\cat{C}\times\cat{D}}{\cat{C}}$ and
$\functor{\pi_2}{\cat{C}\times\cat{D}}{\cat{C}}$ defined in the obvious way.

\subsection{Arrow Category}

The arrow category $\cat{C}^\rightarrow$ has objects which are arrows in \cat{C}
and its arrows are commutative squares in \cat{C}.

For example, given $f,\;f',\;g_1,\;g_2 \in \cat{C}$, there is an arrow $\tfarr{g
}{f}{f'} \in \cat{C}^\rightarrow$ such that $g = (g_1,\;g_2)$ if the follow
diagram exists in \cat{C}:

$$\begin{tikzcd}
  A \cdr{g_1} \cdd{f} & A' \cdd{f'} \\
  B \cdr{g_2} & B'
\end{tikzcd}$$

Composition acts as you'd expect. Given $h \circ g$, we get the diagram:

$$\begin{tikzcd}
  A \cdr{g_1} \cdd{f} & A' \cdr{h_1} \cdd{f'} & A'' \cdd{f''}  \\
  B \cdr{g_2} & B' \cdr{h_2} & B''
\end{tikzcd}$$

In other words, there is an arrow $f\to f'$ iff $g_2\circ f = f' \circ g_1$. But
what does this mean?

In a monoid, composition means concatenation, and thus over a free monoid of the
alphabet, $f = \text{art}$, $f' = \text{far}$, $g = (\text{t}, \text{f})$
because $\text{f}\circ\text{art} = \text{far}\circ\text{t}$. With the monoid of
addition over the natural numbers, the arrow category is complete; all objects
have infinite arrows between them because there are infinite ways of adding two
numbers to two numbers and having them equate.


\todo{more examples plz}
\todo{what are the functors?}

\subsection{Slice Category}

The slice category $\cat{C}/X$, given $X \in \cat{C}$ is a special case of the
arrow category when $B = B' = X$.

\subsubsection{Principal Ideal}

Awodey gives the example of the slice $\cat{P}/P$ over a poset category for some
$P \in \cat{P}$, that this is isomorphic to the principal ideal
$\downarrow\!(P)$. What the heck does this mean? Well, let's draw the diagram:

$$\begin{tikzcd}
  A \cdrr{f} \cddr{!} & & A' \cddl{!} \\
  & P &
\end{tikzcd}$$

In a poset category, we can consider an arrow to be $\le$, therefore, in
$\cat{P}/P$ there is an arrow $\tfarr{f}{A}{A'}$ whenever $A \le A' \le P$.
Therefore, the principal ideal $\downarrow\!(P)$ is just the subset of the poset
that is $\le P$.

\subsubsection{Slice of a Monoid}

Out of curiosity, let's look at the slice $\;\cat{M}/M$ over some monoidal
category where $M \in \cat{M}$.

$$\begin{tikzcd}
  M \cdrr{f} \cddr{g_1} & & M \cddl{g_2} \\
  & M &
\end{tikzcd}$$

We can pull equations out of this diagram, namely that $g_1 = g_2 \circ f$. The
arrows in $\;\cat{M}/M$ are therefore the elements which can be "decomposed"
into the concatenation of two elements. However, because $\;\cat{M}$ is a
monoidal category, all elements can be decomposed (by factoring out a unit).
Therefore, $\;\cat{M}/M \cong\;\cat{M}$.

\subsubsection{Coslice Category}

We can define the coslice category $X/\cat{C}$, given $X \in \cat{C}$ by looking
at $(\cat{C}/X)^{Op}$. This is obviously a special case of the arrow category
when $A = A' = X$.

The coslice of a poset is its $\uparrow\!(X)$, and the coslice of a monoidal
category is still isomorphic to the category itself.

Awodey points out that $1/\cat{Set}$ is isomorphic to the category of pointed
sets, where each set has a distinguished member called the "point" (recall that
$\text{Maybe}$ is the free pointed set). Arrows in $1/\cat{Set}$ are
homomorphisms which preserve the points.

\subsection{Free Monoids}

Given a set $S = \{s_1, s_2, \dots, s_n \}$ of "letters", define $S^* =
\{\text{words over } S\}$, in other words, every finite sequence of letters in
$S$. The empty word can be denoted as -. $S^*$ forms a monoid under
concatenation with - as its unit.

We have an insertion function $i(s) = s : S \hookrightarrow S^*$. The elements
of $S$ generate $S^*$, because every $s \in S^*$ can be made up of some
concatenation of $s_{i1}s_{i2}\dots s_{in}$.

$S^*$ is said to be the free monoid over $S$ because it generates a monoid even
when $S$ itself is not one.

But we can state this definition more "categorically" by way of a
\textit{universal mapping property}.

The free monoid $M(A) \in \cat{Mon}$ over $A \in \cat{Set}$ is the object with
the property that for any $N \in \cat{Mon}$ the following diagram holds:

$$\begin{tikzcd}
  M(A) \cdrn{!} & N
\end{tikzcd}$$

\begin{theorem}
  $M(A) = A^*$
\begin{proof}
  We can prove the theorem by giving a program in Haskell.

  \begin{code}
foldMap :: Monoid n => (a -> n) -> [a] -> n
foldMap _ []       = mempty
foldMap f (a : as) = f a `mappend` foldMap f as
  \end{code}
\end{proof}
\end{theorem}

Awodey says something about how the monoid of natural numbers under addition is
isomorphic to the free monoid over a single element set. This is obviously true,
but I'm not entirely sure what his broader argument is.

\todo{page 28 -- determine what's going on here}

\subsubsection{Forgetful Functors}

For every structured set $Z$ (of category \cat{Z}) there is an underlying set
$\abs{Z}$, and likewise for every homomorphism $f$ over the structured set,
there is a corresponding homomorphism over sets $\abs{f}$. Therefore we have a
functor $\functor{U}{Z}{Set}$ called the forgetful functor.

\subsection{Free Categories}

Just like how $A^* \in \cat{Mon}$ is the free monoid over $A \in \cat{Set}$,
$Cat(G) \in \cat{Cat}$ is the free category over $G \in \cat{Graph}$ (the
category of directed graphs.)

A $Cat(G)$ can be generated from $G$ by taking every vertex $V \in G$ and making
it an object $V \in Cat(G)$. Every path $P$ made up of a finite sequence of
edges $E_1, E_2, \dots E_n \in G$ (where the target of $E_i$ is the source of
$E_{i+1}$) becomes an arrow in $Cat(G)$. Add the mandatory $\tfarr{1_V}{V}{V}$
identities, and you have yourself a category where composition of arrows is
subsequent traversals of paths in the underlying graph.

There is also \textit{universal mapping property} for free categories, namely
that for any category $\cat{C} \in \cat{Cat}$:

$$\begin{tikzcd}
  Cat(G) \cdrn{!} & \cat{C}
\end{tikzcd}$$

\todo{prove this}

\subsection{More Foundations}

A category is said to be "small" iff its objects and arrows both form sets.
Otherwise it is "large."

This implies that \cat{Set} and any other structured set categories are large,
and that \cat{Cat} is itself only the category of small categories -- and thus
does not contain itself.

However, a category \cat{C} can be said to be "locally small" iff for any
objects $X, Y \in \cat{C}$, the set of arrows $\{f \in \cat{C}\;|\;f : X \to Y
\}$ is a set. \cat{Set} is locally small because there is a set of all functions
from $X \to Y$, $Y^X$. Likewise, all other structured sets share this property,
since they are necessarily more restrictive than \cat{Set}.

Awodey asks whether \cat{Cat} is locally small. My intuition is yes, because all
of its objects are themselves small categories; therefore the collection of
functors from one small category to another must form a set.

\subsection{Exercises}

\begin{exercise}
  The objects of \cat{Rel} are sets, and an arrow \tfarr{f}{A}{B} is a relation
  from $A$ to $B$, that is, $f \subseteq A \times B$. The identity relation
  $\{\langle a, a\rangle \in A \times A \;|\; a \in A\}$ is the identity arrow
  on a set $A$.  Composition in \cat{Rel} is to be given by:

  $$
  g \circ f = \{\langle a, c\rangle \in A \times C \;|\; \exists b. \langle a,b
  \rangle \in f \;\&\; \langle b, c \rangle \in g\}
  $$

  Show that \cat{Rel} is a category.

  \begin{proof}
    We need to show that $\langle a, a\rangle$ is an identity, and that $f \circ
    (g \circ h) = (f \circ g) \circ h$.

    Given \tfarr{f}{A}{B}, we can show that $1_a = \langle a, a \rangle$ is an
    identity:

    \begin{align*}
      f \circ 1_a &=
  \{\langle a, b\rangle \in A \times B \;|\; \exists x. \langle x,x
  \rangle \in 1_a \;\&\; \langle x, b \rangle \in f\} \\
      &= \{\langle a, b\rangle \in A \times B \;|\; \langle a,a
  \rangle \in 1_a \;\&\; \langle a, b \rangle \in f\} \\
      &= \{\langle a, b\rangle \in A \times B \;|\;
  \langle a, b \rangle \in f\} \\
      &= f
    \end{align*}

    The proof that $1_b$ is a left identity proceeds in exactly the same way.
    Therefore, the identity relation is in fact an identity on the arrows.

    Finally, we need to show that composition is associative. This proceeds
    immediately from the existential in the definition of composition which
    asserts an equality between the (set) codomain of $f$ and the domain of $g$.
    Because equality is associative, composition in \cat{Rel} must too be.
  \end{proof}
\end{exercise}

\begin{exercise}
  Determine which of the following isomorphisms hold:

  \begin{enumerate}
    \item{$\cat{Rel} \cong \cat{Rel}^{op}$}
    \item{$\cat{Set} \cong \cat{Set}^{op}$}
    \item{For a fixed set $X$ with powerset $P(X)$, as poset categories $P(X)
      \cong P(x)^{op}$ (the arrows in $P(X)$ are subset inclusions $A \subseteq
      B$ for $A, B \subseteq X$)}
  \end{enumerate}

  \begin{proof}
    \quad
    \begin{enumerate}
    \item{$\cat{Rel} \cong \cat{Rel}^{op}$ because arrows compose due to an
      existential equality. Since equality is dual to itself, these two
      categories must be isomorphic.}
    \item{$\cat{Set} \not\cong \cat{Set}^{op}$ because in the dual, initial
      objects in \cat{Set} are mapped to terminal objects in $\cat{Set}^{op}$.
        Since isomorphisms need to preserve initial objects (and other
        interesting features of a category), these two categories are not
        isomorphic.}
    \item{\todo{Apparently this is in fact isomorphic to its dual, but I don't
      know why because it seems like the \cat{Set} argument should apply here as
        well.}}
    \end{enumerate}
  \end{proof}
\end{exercise}

\section{Abstract Structures}

\subsection{Epis and Monos}

A monomorphism (mono) is any morphism \tfarr[\mono]{f}{A}{B} such that for any
\tfarr{g, h}{C}{A} the following is true: $fg = fh \implies g = h$. In other
words, a mono is always left-cancelable.

$$\begin{tikzcd}
  C \arrow[shift right=2,swap]{r}{h} \arrow[shift left=1]{r}{g} & A \cdr{f} & B
\end{tikzcd}$$

Dually, an epimorphism (epi) is a morphism \tfarr[\epi]{f}{A}{B} such that for
any \tfarr{i, j}{B}{D} the following is true: $if = jf \implies i = j$. An epi
is always right-cancelable.

$$\begin{tikzcd}
  A \cdr{f} & B \arrow[shift right=2,swap]{r}{j} \arrow[shift left=1]{r}{i} & D
\end{tikzcd}$$

In \cat{Set}, monos correspond to injective functions, and epis are surjective
functions. Furthermore, in \cat{Set} a function which is monic and epic is an
iso, but this is \textbf{not true} in general.

\begin{theorem}
  In a poset \;\cat{P}, every arrow $p \leq q$ is both monic and epic.
  \begin{proof}
    In a poset, there is exactly one arrow between any two objects. Therefore if
    two arrows are equal after composing with a third, they must have been equal
    to begin with.
  \end{proof}
\end{theorem}

\begin{theorem}
  Every iso is a mono and an epi.
  \begin{proof}
    Given an iso \tfarr{m}{A}{B} with inverse \tfarr{e}{B}{A}:
    \begin{align*}
      mx &= my \\
      emx &= emy \\
      1_B x &= 1_B y \\
      x &= y
    \end{align*}

    And dually to show epicness.
  \end{proof}
\end{theorem}

\subsection{Initial and Terminal Objects}

An object $0 \in \cat{C}$ is said to be initial if for every other object $X \in
\cat{C}$ there is a unique arrow \tfarr{!}{0}{X}. Dually, a terminal object is
one which has a unique arrow coming into it from every object in the category.

\begin{theorem}
  Initial and terminal objects are unique up to isomorphism.
  \begin{proof}
    Assume we have two initial objects, $X$ and $Y$. In order to be initial,
    they must have arrows \tfarr{y}{X}{Y} and \tfarr{x}{Y}{X}. Furthermore, we
    know that there is exactly one arrow from an initial object to any other --
    including itself. By the category laws, this unique arrow must be 1, which
    makes the following diagram commute:

$$\begin{tikzcd}
  X \cdr{y} \cddr{1_X} & Y \cdd{x} \arrow{dr}{1_Y} \\
  & X \cdr{y} & Y
\end{tikzcd}$$

    This argument dualizes to terminal objects.

  \end{proof}
\end{theorem}

\end{document}
